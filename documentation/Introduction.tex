
\chapter{Introduction}

\beavtex (the Beaver's \LaTeX{}) is a \LaTeX{} class designed specifically
to format theses and dissertations at Oregon State University. It
was created almost completely from the ground up as a fully functional
\LaTeX{} class. The original class was found to be inadequate, as
it was mainly a complex set of templates and files and borrowed classes
from other universities that was very hard to use. Many intrepid explorers
tried to fix it bit by bit and send it to others in an improved form.
I decided to essentially scrap everything and start from scratch,
borrowing specific macros and definitions (mostly from the Table of
Contents down, in the source file).

This chapter introduces \beavtex and gives some hints on installation.
The following chapters detail \beavtexnospace 's commands/internals,
and introduce using \beavtex in \LyX{}. I hope you enjoy using \beavtex,
and find that it helps you in writing your thesis. Please send me
any ideas you may have about improving it.

\bigskip{}
\noindent --John Metta

\noindent \texttt{<john.pennington@lifetime.oregonstate.edu>}


\section{\LaTeX{} and \LyX{}}

Some quick information first, for those who may not be aware. \LaTeX{}
has a graphical front-end called \LyX{}. Depending on your field of
study, learning \LaTeX{} may be a good idea; however, it's quite possible
to get along with \LyX{} without knowing a great deal of \LaTeX{}.


\subsection{Getting and Learning \LaTeX{}}

If you are reading this, you are either aware of, or curious about,
the beauty of using \LaTeX{} to format documents. For windows users
who may have gotten a hold of this document and want to try \LaTeX{},
the distribution you should look for is called MiK\TeX{}. \emph{MiK\TeX{}
is Free.}

My suggestion, for those who are not total \LaTeX{} gurus, is to get
a copy of the User's Guide by~\citet{lamport.94:LaTeX}. It is a
fast reference for nearly all of the commands you will need in your
day-to-day \LaTeX{} life. Another source, for those who want to learn
\TeX{} programming, is The \TeX{}book~\citep{knuth.84:TeXbook}.
It is the definitive guide to \TeX{} as a programming language.


\subsection{Using \LaTeX{} Without Learning It}

Even if you know \LaTeX{}, you may not be aware of \LyX{} (\texttt{http://www.lyx.org}).
\LyX{} is a graphical user interface to \LaTeX{}. If you are a Micro\$oft
user, think of it as a {}``Word for \LaTeX{}''. It's available for
Linux, Mac and Windows, and makes using \LaTeX{} possible, and even
easy, without knowing anything about \LaTeX{} programming.

For those who are \LaTeX{} hackers, it basically makes your life a
lot easier, because you can use it as a front end, but also have the
ability to add \LaTeX{} code whereever/whenever you want. \LyX{} will
not get in your way.

\LyX{} is essentially the only thing I use to write with, since everything
can be done in it. It's a wonderful thing, but I'm probably biased.
In any case, if you are not using \LyX{}, I suggest you look into
it.


\section{Installing \beavtex}

I've decided not to distribute \beavtex in the standard CTAN manner,
mainly because it will be provided specifically to students at OSU,
and all others would find it essentially useless but for the possibility
of using it to build another class. 

This source distribution includes 4 folders: 

\begin{lyxlist}{00.00.0000}
\item [\texttt{archive}]This folder is a collection of various versions
and assorted files of the original osuthesis class. These are provided,
in no particular organizational structure, mainly to satisfy the curiosity
of \TeX{}/\LaTeX{} hackers. You can safely discard this folder.
\item [\texttt{documentation}]Where the \LaTeX{} version of the documentation,
as well as the most recent version of the Oregon State Graduate School
style guide, resides. Do with this what you wish.
\item [\texttt{latex}]Contains the beavtex folder, copy this folder in
its entirety (it contains only beavtex.cls) to the appropriate place
in your \LaTeX{} distribution. After you copy these files, you will
have to run texhash as the superuser. Windows users should consult
their MiK\TeX{} documentation to see about installing classes.
\item [\texttt{lyx}]Contains the \LyX{} layout definitions. \texttt{beavtexchapter.layout}
is the layout used for individual chapters and sections. \texttt{beavtexmaster.layout}
is the main file layout. Copy these layout files to the layouts folder
of your \LyX{} program. Run \LyX{}, go to \textsf{\underbar{E}}\textsf{dit\lyxarrow{}Reconfigure,}
then restart \LyX{}.
\end{lyxlist}
That should do it. With this little effort, you'll be on your way
to spending time writing your thesis, rather than formatting it.


\section{Documentation as Example}

You can use \beavtexnospace 's documentation as an example of how
to build a thesis. One way is to look into the \LaTeX{} file \texttt{beavtex.tex}.
This is the main file for the documentation, and includes the other
files so that you can see how to include individual chapters as seperate
files. The other way is to use \LyX{} and open the file \texttt{beavtex.lyx}.
The thesis was actually written using \LyX{}, and exported into \LaTeX{}
and PDF formats from there.


\subsection{Typesetting the Name}

You can typeset the name \beavtex within your document by using the
\LaTeX{} command \texttt{\textbackslash{}beavtex}. This will automatically
insert a trailing space, so if you want to say something like {}``\beavtexnospace
's cool layout options rule,'' use the nospace version as so

\begin{lyxcode}
\textbackslash{}beavtexnospace~'s~cool~layout~options~rule
\end{lyxcode}
Honestly, this is only available for purely egotistical reasons. I
mean, there's absolutely no reason that you would \emph{need} this
functionality, unless you were going to use it to acknowledge \beavtex
as\ldots{}say\ldots{}the single most important thing you've used
to complete your thesis.

\begin{comment}
\bibliographystyle{jawra}
\bibliography{bibfile}

\end{comment}

